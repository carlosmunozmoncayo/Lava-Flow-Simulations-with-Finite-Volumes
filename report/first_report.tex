\documentclass[12pt]{extarticle}
\usepackage[utf8]{inputenc}
\usepackage{cite}
\usepackage{amsmath}

\title{AMCS 333 Project Proposal}
\author{Carlos Muñoz}
\date{\today}

\begin{document}

\maketitle

\section{Problem}
Given the large number of people living near active volcanoes worldwide, it is of prime importance to study the nature and evolution of volcanic eruptions. Lava flows propagation is particularly  important in order to develop effective construction and hazard mitigation strategies. Given the variety of physical phenomena involved in lava flow such as volcano topography, eruptive and thermal conditions, and rheological laws, there is not a single dominant approach for its simulation and forecasting. Some of the most popular numerical models are based on cellular automata, stochastic processes, and depth-averaged partial differential equations, being the latter the approach of interest in this project \cite{cordonnier2016benchmarking}.


\section{Method}

We consider the model proposed by Costa and Macedonio \cite{costa2005numerical} based on the two-dimensional shallow water equations with source terms, \eqref{eq: 1}-\eqref{eq: 4}. Here an homogeneous incompressible fluid is assumed, and the depth of the lava $h$, the particle velocities in the x and y components, $U$ and $V$, and the temperature $T$, are assumed to be the average across any vertical column of fluid. The thermal balance equation \eqref{eq: 4} is added given the temperature dependence of the viscosity. $\beta_ij$, $\mathcal{E}$, $\mathcal{W}$, and $\mathcal{K}$ are semiempirical parameters, $T_c$, $T_{env}$, $T_{r}$ represent the temperature of the ground, environment, and lava at the vent respectively. $\gamma(T)$ is a function that compresses wall friction and the effect of heat in the flow velocity.

\begin{align}
\frac{\partial h}{\partial t}+\frac{\partial(U h)}{\partial x}+\frac{\partial(V h)}{\partial y}&=0 \label{eq: 1}\\
\frac{\partial(U h)}{\partial t}+\frac{\partial\left(\beta_{x x} U^{2} h+g h^{2} / 2\right)}{\partial x}+\frac{\partial\left(\beta_{y x} U V h\right)}{\partial y}&=-g h \frac{\partial H}{\partial x}-\gamma U \label{eq: 2}\\
\frac{\partial(V h)}{\partial t}+\frac{\partial\left(\beta_{x y} U V h\right)}{\partial x}+\frac{\partial\left(\beta_{y y} V^{2} h+g h^{2} / 2\right)}{\partial y}&=-g h \frac{\partial H}{\partial y}-\gamma V \label{eq: 3}\\
\frac{\partial(T h)}{\partial t}+\frac{\partial\left(\beta_{T x} U T h\right)}{\partial x}+\frac{\partial\left(\beta_{T y} V T h\right)}{\partial y}&=-\mathcal{E}\left(T^{4}-T_{e n v}^{4}\right)+ \label{eq: 4}\\
\quad-\mathcal{W}\left(T-T_{e n v}\right)-\mathcal{H}\left(T-T_{c}\right)+\mathcal{K}\left(U^{2}+V^{2}\right) &\exp \left[-b\left(T-T_{r}\right)\right] \notag
\end{align}
 

\section{Goals}

The main objective of the proposed project is to reproduce the numerical method for simulation of lava flows presented in \cite{costa2005numerical} using PyClaw. For this purpose an appropriate Riemann solver for the homogeneous system must be implemented, since the original method uses Godunov splitting to handle the source terms. Given the stiffness of the latter, an implicit time integration method must also be used. A difficulty that will be faced in this process is the presence of a wet-dry front at every time. \\

The performance of the method should be tested first with problems with known analytical solutions and then in a real world problem. To obtain the data used in \cite{costa2005numerical} would allow us to make a comparison with the original implementation. Also, several volcanic eruptions in recent years have been documented using top-notch technology and if it was accessible, it would be interesting to validate the model with this data.


\bibliographystyle{plain}
\bibliography{M335}

\end{document}
